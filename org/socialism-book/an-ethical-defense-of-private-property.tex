% Created 2023-08-29 Tue 17:39
% Intended LaTeX compiler: pdflatex
\documentclass[11pt]{article}
\usepackage[utf8]{inputenc}
\usepackage[T1]{fontenc}
\usepackage{graphicx}
\usepackage{longtable}
\usepackage{wrapfig}
\usepackage{rotating}
\usepackage[normalem]{ulem}
\usepackage{amsmath}
\usepackage{amssymb}
\usepackage{capt-of}
\usepackage{hyperref}
\author{LiquidZulu}
\date{}
\title{An Ethical Defense of Private Property}
\hypersetup{
 pdfauthor={LiquidZulu},
 pdftitle={An Ethical Defense of Private Property},
 pdfkeywords={},
 pdfsubject={},
 pdfcreator={Emacs 28.2 (Org mode 9.5.5)}, 
 pdflang={English}}
\begin{document}

\maketitle

\section{The Problem of Law}
\label{sec:org8b8c025}
Ethics is the science whose concern is the identification of those principles which contrast right from wrong in the realm of human action. Ethics--like all sciences--does not stand on it's own---any given ethical theory must rest upon the general theory of knowledge, \emph{epistemology}, and more fundamentally on the theory of the nature of the universe as a whole, \emph{metaphysics}. This section will not be explaining ethical theory as such, rather the focus is placed upon a specific subset of ethics called \emph{law}.

To see the problem that law studies picture a desert island inhabited by Crusoe and Friday. One day, Crusoe finds a branch poking out of a tree that he expects would make for a perfect spear to fish with. Crusoe takes this branch out of the tree, processes it into a spear, and heads to the ocean to get his dinner. On his way there, however, Friday sees this spear that Crusoe is carrying and imagines that it would suit him in stoking his fire. Friday decides to try and take this spear from Crusoe for this purpose.

What we notice in this scenario is that it is not possible for \emph{both} Crusoe \emph{and} Friday to get what they want; if Friday is able to successfully wrestle the stick from Crusoe then his end is satisfied but the end of Crusoe is not, and vice versa for if Crusoe can repel Friday's attempts. In short, one mans action with the spear excludes a whole host of alternative actions that other men may wish to use the spear in. This is the economic principle of \emph{scarcity}.\footnote{The term \emph{rivalrousness} may also be used.}

Given scarcity it is possible for men to come into conflicts, a conflict being defined as mutually-exclusive actions. So, in the above scenario when Friday decides to try and take the spear from Crusoe we say that there is a conflict over that spear; one man's use of the spear excludes the other from using it---the actions are mutually-exclusive. This is the issue that law studies, namely we have this problem of conflicts that can arise due to the fact that goods are scarce, law answers how we should deal with these conflicts.

There are three basic answers to this question:\footnote{This is a modified form of Murray Rothbard's \emph{argumentum e contrario} for self-ownership rights: idem., ``Interpersonal Relations: Ownership and Aggression,'' in idem., \emph{The Ethics of Liberty}, p. 45}
\begin{enumerate}
\item that conflicts should be avoided (the non-aggression principle);
\item that conflicts should be avoided under certain circumstances, but not always (mixed law), or;
\item that conflicts should not be avoided (the law of the jungle).
\end{enumerate}

\section{An Analysis of Competing Legal Theories}
\label{sec:orgb2f8e30}
\subsection{The Law of the Jungle}
\label{sec:orgb28d723}
First let's consider the law of the jungle---what would a universal acceptance of conflict-engendering norms look like? If Crusoe were to take a stick from nature and try to use it for spearfishing, he would not be able to complain under this system if Friday came along and took that stick from him, because Crusoe could not claim the stick as his \emph{property}, the property right would rather travel from person to person always being attached to the item itself. In other words, on such a view there would be no distinction whatsoever between \emph{ownership} and \emph{possession}. We would have--in the words of Stephan Kinsella\footnote{Stephan Kinsella, \emph{Thoughts on the Latecomer and Homesteading Ideas; or, why the very idea of ``ownership'' implies that only libertarian principles are justifiable}.}--a ``mere possessor'' ethic.

The issue with this view is that ownership--which we can define as the \emph{right} to possess a given scarce good--is necessarily distinct from possession. If there is some dispute between \(A\) and \(B\) over who should be the one to control a given property, then \emph{both} \(A\) and \(B\) must pre-suppose this to be the case. \(A\) is asserting that though \(B\) might be able to actually obtain control, it would nevertheless be the case that \(A\) \emph{should} be the one to control it, and similarly \(B\) is asserting that though \(A\) might be able to actually obtain control, it would nevertheless be the case that \(\textit{\textbf{B}}\) should be the one to actually control it.

So this mere possessor ethic, which the law of the jungle asserts, would require a conflation of the concepts \emph{ownership} and \emph{possession}, that is to say that the law of the jungle is the assertion that ownership rights are acquired by the mere act of taking a given good from someone else---if \(A\) has a stick and \(B\) takes that stick from him, then the jungle-jurist says that \(C\) could come along and take the stick from \(B\) and then become the owner, i.e. that whomever is in possession of the stick is in fact it's owner. But, how exactly is a person to assert this jungle-law view in defense of their actions? We saw above that both \(A\) and \(B\) must pre-suppose the distinction between ownership and possession---they are saying that \emph{they} should control the item which implies that they have a \emph{right} to exclude other people from using it.\footnote{Recall the point about scarcity; if they have a right to control the item in question this control will necessarily exclude others from using it.}

We have this real problem of conflicts that we are trying to resolve and the jungle law view is simply incommunicable by the fact that it is a contradiction to even assert---the instant a man tries to defend his conduct by asserting that conflicts should not be avoided and that rights are illusory he necessarily asserts that conflicts should be avoided (when initiated against him) and that rights are real (when the thief is facing a counter-attack).\footnote{For more on this see: N. Stephan Kinsella, ``Rights-Skepticism,'' in idem. \emph{Dialogical Arguments for Libertarian Rights}; LiquidZulu, ``The Contradiction of Rights-Scepticism,'' in idem., ``2. The Non-Aggression Principle,'' in idem., \emph{The Fundamentals of Libertarian Ethics}, \url{https://liquidzulu.github.io/the-nap}} So this man would be left with only the option of sealing up his lips and making no defense, living as an animal-beast ruled by whatever whims he feels at the moment, with no concern for whether his conduct is rationally defensible.

So at best the jungle-law ethic reduces into whim-worship, but recall above that ethics itself rests upon earlier conclusions in metaphysics and epistemology, so upon what metaphysical and/or epistemic premises does such an ethic rest? Fundamentally, we have the question of ``how should we be dealing with conflicts, what is criminal?'' and the jungle-jurist asserts: ``who gives a damn? Might makes right; live by your arbitrary whims.'' What this means, if taken as a serious ethical proposal, is that whims are a genuine source of knowledge, i.e. this is not only a whim-ethic, but a whim-epistemology---it all boils down to ``I think this is true because I feel like it is;'' ``I should take this spear because I feel like I should.''

But of course, epistemology does not stand on it's own, it is not primary in philosophy; rather a given epistemology rests on prior metaphysical premises. So on what metaphysical premises does this whim-epistemology rest? What is really being said here is that if you simply \emph{think} something to be the case hard enough then it \emph{is} the case; that your whims, your thoughts, your consciousness is the basis of reality. That existence conforms to your consciousness, rather than the other way around. This is the fallacy of the primacy of consciousness. This view of the law of the jungle or any other whim-based theory does and must rely on the premise that consciousness--mere thoughts--have metaphysical primacy over existence.

The reason that this view is fallacious and an inversion of the correct way of doing things is that the concept of ``consciousness'' \emph{requires} a \emph{prior} concept of \emph{existence}. It is simply meaningless to speak of \emph{consciousness} as floating on its own; to be \emph{conscious} is to be conscious \emph{of something}. The whim-epistemologists steal the concept of consciousness. This is akin to those who assert such things as ``property is theft'':
\begin{quote}
“Theft” is a concept that logically and genetically depends on the antecedent concept of “rightfully owned property”—and refers to the act of taking that property without the owner’s consent. If no property is rightfully owned, that is, if nothing is property, there can be no such concept as “theft.” Thus, the statement “All property is theft” has an internal contradiction: to use the concept “theft” while denying the validity of the concept of “property,” is to use “theft” as a concept to which one has no logical right—that is, as a stolen concept.\footnote{Nathaniel Branden, \emph{The Fallacy of The Stolen Concept}}
\end{quote}

So to bring this into clarity with the case of our whim-epistemologists, consciousness relies on the antecedent concept of existence---there is \(\text{no\textit{thing}}\) to be conscious \emph{of} without \emph{things} existing in the first place. Therefore, this notion of a law of the jungle wipes itself out---it is a self-defeating idea, completely unworthy of any consideration at all.

\subsection{Mixed Law}
\label{sec:orga1724b8}
\subsubsection{Reduction to The Primacy of Consciousness}
\label{sec:orga34c07e}
Next, let's consider the ``mixed law'' system(s); i.e. that conflicts should be avoided under certain circumstances, but not always. First any mixed-law system that can be reduced to ``we must aggress\footnote{Here ``aggression'' is defined as the initiation of conflict.} in these arbitrary situations'' is refuted by the above reasoning against the law of the jungle.

Allow me to now quickly introduce a number of different proposals for a mixed-law system such that I may draw your attention to a common principle among them all:
\begin{enumerate}
\item consequentialism---he whose victory would yield the best outcome is he who should win the conflict at hand;\footnote{Utilitarianism would be a sub-category, where ``best outcome'' means ``the outcome of maximal utility.'' Utilitarianism specifically can be shown to be false on purely economic grounds in its use of a category error with the determination of the data type of ``utility,'' see: Kenneth A. Zahringer (2011), ``Cardinal Utility: It's Worse Than You Thought,'' \emph{Mises Daily}}
\item racism---he who is fighting for the interests of the preferred race is he who should win the conflict at hand;
\item Marxism---he who is fighting for the interests of the proletariat is he who should win the conflict at hand;
\item primitivism---he who is fighting for apocalypse\footnote{Apocalypse here is meant as a break-down in the capital structure, which is what primitivists such as Ted Kaczynski advocate for. For more on this see: LiquidZulu, \emph{Primitivism is an Apocalyptic Ideology}.} is he who should win the conflict at hand;
\item monarchism---he who is deemed to be the proper victor by the monarch is he who should win the conflict at hand;
\item democratism---he who is deemed to be the proper victor by majority opinion is he who should win the conflict at hand;
\item Rawlsianism---he who is deemed to be the proper victor by a party situated behind a veil of ignorance is he who should win the conflict at hand,\footnote{For more on this in particular see: Hans-Hermann Hoppe, Introduction to Murray Rothbard, \emph{The Ethics of Liberty}, second edition.} and;
\item imperialism---he who is deemed to be an ally by the military leadership of the preferred country is he who should win the conflict at hand.
\end{enumerate}

You will notice that on their face these ideologies fall under two categories: (1) the class-based, i.e. ``he who is part of the preferred class is he who should win the conflict at hand,''\footnote{This would include 1-4; consequentialism has the class of detrimental individuals as against the class of beneficial individuals, primitivism has the class of producers as against the class of anti-producers, racism and marxism are obvious.} and (2) the whim-based, i.e. ``he who is deemed to be the proper victor by X is he who should win the conflict at hand.'' It should be clear why the latter would fall under the same reasoning as used against the law of the jungle; thus I shall focus my efforts on those class-based mixed-law systems.

Any form of class-based law is an ethic in the form: one rule for class \(A\) and another for class \(A^\complement\).\footnote{Here \(A^\complement\) is the \emph{complement} of \(A\), i.e. the set of all people who aren't in the class \(A\).} But by what possible means could one derive that one ethic applies to \(A\) and another \emph{incompatible}\footnote{If the ethics were compatible this would then not be one rule for \(A\) and another for \(A^\complement\), it would rather be the same fundamental principle applying to both groups, and would therefore not be a class-based ethic.} ethic applies to \(A^\complement\)? Surely such an ethic could not be derived from the nature of man as such, because if it were then we would have a universal principle, not one that applies only to a particular subset of humanity. Therefore, such an ethic must be arbitrarily particularised---we have an arbitrary distinction which forms a class of humans and a class of sub-humans, we do not here have a rational ethic for \emph{man}. This particularisation then falls back into the primacy of consciousness and therefore fails.

However, the above reasoning is assuming that this individual accepts the metaphysical equality of man, as without such an equality it would not be the case that man as such has a particular nature which implies a certain ethics but rather there might be a proletarian class with their own proletarian logic as against a bourgeoisie class with their own bourgeoisie logic. This is polylogist thesis, first identified by Ludwig von Mises. Polylogism, therefore, must be analysed before we can proceed. First, the polylogist requirement that different groups have different logics rests upon the prior assumption that the laws of logic are subjective, rather than objective\footnote{I use these terms in their metaphysical sense, i.e. objective meaning mind-independent and subjective meaning mind-dependent.}---that logic is not imposed upon man as a necessary requirement of validating his beliefs by the universe, but rather that man himself projects out such rules onto the universe.

This is, again, primacy of consciousness---the starting point for the polylogist is not an observation of reality and derivation therefrom, but rather a deadly retreat into their own minds. When we have the proper starting point of existence we have it that the very first thing you can say metaphysically is that existence exists.\footnote{On this see: Leonard Peikoff, ``Existence, Consciousness, and Identity as the Basic Axioms,'' in idem., \emph{Objectivism: The Philosophy of Ayn Rand}.} \emph{From this} we have it that we are conscious \emph{of} existence. And of course, to be conscious of existence means that you are conscious of \(\tert{some\textit{thing}}\) that exists, implicit in this is the law of identity: that which is is what it is, \(A\) is \(A\).

\begin{quote}
If something \emph{exists}, then \emph{something} exists; and if there is a \emph{something}, then there \emph{is} a something.\footnote{ibid.}
\end{quote}

\subsubsection{The Argument from Argument}
\label{sec:org7a9a7d2}
On top of this, there exists a built-in self-destruct for any mixed law ethic, in the form of Hans-Hermann Hoppe's argument from argument.\footnote{On this, see: Hans-Hermann Hoppe, ``The Ethical Justification of Capitalism and Why Socialism is Morally Indefensible,'' in idem., \emph{A Theory of Socialism and Capitalism}; N. Stephan Kinsella, \emph{Dialogical Arguments for Libertarian Rights}; and Frank van Dun, \emph{Argumentation Ethics and the Philosophy of Freedom}.} The basic idea behind this attack is found in noticing that there exists an inescapable inconsistency when it comes to arguing in favour of aggression, borne from the fact that argumentation is and must be a conflict-free interaction. When people have some dispute and they choose to argue about it, they are doing the exact opposite of fighting over the dispute. That is, if Crusoe and Friday have a disagreement over how to use a spear, then each party sitting down and giving arguments as to why their use should go forth is quite distinct to each party launching missiles and trying to stab the other to death in order that their use may go forth unimpeded.

That is, simply by arguing about property rights, you must pre-suppose libertarian non-aggression in your act of peacefully attempting to resolve the disagreement. For our above dispute between Crusoe and Friday, if Friday is trying to convince Crusoe that the proper use of the spear is to violate Crusoe's bodily autonomy, then he finds himself in a practical contradiction,\footnote{The terms performative contradiction and dialectic contradiction are also often used here; though dialectic contradiction is really a subset of performative contradictions in general.} namely he is respecting Crusoe's bodily autonomy and trying to achieve consent from Crusoe by his act of arguing, whilst he is explicitly rejecting that Crusoe's consent is required in the first place. To escape this contradiction, Friday has two options: first, he can stop arguing and go back to fighting over it, or second, he can drop his claim that Crusoe's bodily autonomy should be violated. In this second case the mixed law or jungle ethic has trivially dropped out of rational consideration, and in the first case we have it that Friday has turned himself into an animal-beast governed only by whim---which makes his ethic irrational still.

What this argument does is highlight an implicit notion we have that such jungle ethics are irrational and brutish; namely that it is simply inconsistent and hypocritical for a person to even try to assert them in an argument---that if they truly believe in their murderous creeds then why the hell aren't they living by them? This is the same inconsistency that is present in those anti-human environmentalists who advocate that fewer humans be born because of how evil and destructive man is---the proper course with such an individual is to point out that if they are correct then they should simply kill themselves and reduce the problem by one.\footnote{I believe that this phrasing is from Hans Hoppe but I do not recall from where.}

\subsection{Private Property Anarchism}
\label{sec:org3f84122}
So, we have it that of our alternative legal systems both the law of the jungle and any sort of mixed law fall into complete irrationality, leaving us with total adherence to the Non-Aggression Principle as the only alternative. It is proper at this point for me to briefly indicate the sort of legal theory that this implies, but as this is only a \emph{defense} of private property, I shall not exhaust said legal theory.\footnote{If you want to know more about it see: LiquidZulu, \emph{The Fundamentals of Libertarian Ethics}, \url{https://liquidzulu.github.io/libertarian-ethics}; Murray Rothbard, \emph{The Ethics of Liberty}; and/or Kris Borer, \emph{The Ethics of Anarcho-Capitalism}.}

First, under this theory you will recall that property rights--that is, rights of exclusive control over scarce resources--are not acquired by arbitrary fiat whether divine, or social, or personal. Rather property rights are acquired by a process of \emph{homesteading}, that is by taking up initial control over the good in question. In our case of Crusoe taking the spear from nature prior to anyone else coming into contact with it we saw that there is no conflict until the moment that Friday attempts to re-possess Crusoe's spear---taking things from nature is a legitimate action. We saw that it is Friday who is committing the crime, that is any latecomers to the spear cannot justifiably control it, only the firstcomer--i.e. the homesteader--can, any latecomer is initiating a conflict when they attempt to re-possess the good in question.

We can move simply from this understanding of rational property rights as conflict-avoiding norms that are obtained by the firstcomer into an insight into the nature of a legitimate homesteading claim, namely that the borders of said homesteaded property must be objective and intersubjectively ascertainable.\footnote{The latter condition is subsumed within the first under an Objectivist epistemology, but this is not relevant here and splitting these two up is easier to understand. Objectivists in the audience will simply have to grit their teeth and trust that I am not trying to sneak Kant through the back door. If you are interested see: Ayn Rand, \emph{Introduction to Objectivist Epistemology}.} This fact becomes clear when we break it down into its constituent parts: first, property borders must be objective---that is they must adhere to existence as against the arbitrary content of ones consciousness. This is simply another way of explaining that property rights do not derive by fiat: arbitrary declarations as the basis of property claims simply cannot avoid conflicts and such a fiat-law rests on untenable metaphysical ground. We have it that a rational legal system must be universal, not arbitrarily particularised, thus we cannot have subjective property borders as subjective property borders may freely contradict each other, leaving the theory in ruin.

Now let me turn to the criterion that property borders must be intersubjectively ascertainable---what this means is that they are “public” and can be seen or in some other way perceived by third parties. The reason why rational property borders must be intersubjectively ascertainable is that if they are not they cannot serve to avoid conflicts, and property rights are conflict avoiding norms. Consider a non-intersubjectively ascertainable property border such as a mere verbal decree---Robinson Crusoe decides that he wants to own the Moon so he simply shouts to the forest that he now owns the Moon. Clearly this verbal decree cannot serve to avoid conflicts, when an astronaut is approaching the Moon he has no means of discovering Crusoe's supposed property right in it, thus Crusoe has not actually engaged in an act of homesteading---he has not erected an objective, intersubjectively ascertainable border.

\begin{quote}
[\ldots{}] no one could ever deny that norms for determining the ownership of scarce goods are useful for allowing conflict-free exploitation of such resources. But, as Hoppe points out, there are only two fundamental alternatives for acquiring unowned property: (1) by doing something with things with which no one else had ever done anything before, that is, the Lockean concept of mixing of labour, or homesteading; or (2) simply by verbal declaration or decree. However, a rule that allows property to be owned by mere verbal declaration cannot serve to avoid conflicts, since any number of people could at any time assert conflicting claims of ownership of a particular scarce resource. Only the first alternative, that of Lockean homesteading, establishes an objective [\ldots{}] link between a particular person and a particular scarce resource, and thus no one can deny the Lockean right to homestead unowned resources.\footnote{N. Stephan Kinsella, ``Argumentation Ethics,'' in idem. \emph{Dialogical Arguments for Libertarian Rights}.}
\end{quote}

A further fundamental implication that must here be made explicit is that these property rights are \emph{exclusive} both in the sense that they exclude non-owners,\footnote{For more on this beyond what has been explained here see N. Stephan Kinsella's discussion of Anthony de Jasay's ``let exclusion stand'' principle in \emph{Thoughts on the Latecomer and Homesteading Ideas; or, why the very idea of “ownership” implies that only libertarian principles are justifiable}: \url{https://www.stephankinsella.com/2007/08/thoughts-on-the-latecomer-and-homesteading-ideas/}} and also in the sense that they must be held \emph{exclusively} by a \emph{single} person. That is, no rational theory of property rights can include \emph{group} or \emph{collective} property rights, they must be \emph{private} property rights. Consider a set of people, \(\{A,\cdots,Z\}\), who each commonly own a stick. What is to be done about a conflict over the use of this stick between \(A\) and \(B\)? There are two possibilities, either \(A\) is said to be the just victor, or \(B\) is. If \(A\), then he owns the stick and \(B\) does not, if \(B\) then he owns the stick, and \(A\) does not. But both options contradict the presumption that every member in the set owned the stick, therefore group ownership simply cannot occur.

Allow me to go over some supposed solutions to this conundrum, the first of which is the democratic one. Essentially, have all members within the set vote to determine who the just victor is---still, any who lost the vote did not own the stick, as their desired control was considered unjust. Also consider the set which only consists of \(A\) and \(B\), what vote could possibly be conducted between these men which would not come out as \(A\) in favour of \(A\) and \(B\) in favour of \(B\)? If \(B\) voted for \(A\) or vice versa there would be no conflict, and law studies only those set of situations where there is conflict rather than those where men are in harmonious agreement about how things should be done.

The next proposal for a solution comes from Roderick Long, he sates:\footnote{Roderick T. Long (1996), “The Ethical Argument,” in idem. \emph{In Defense of Public Space}.}
\begin{quote}
On the libertarian view, we have a right to the fruit of our labor, and we also have a right to what people freely give us. Public property can arise in both these ways.

Consider a village near a lake. It is common for the villagers to walk down to the lake to go fishing. In the early days of the community it’s hard to get to the lake because of all the bushes and fallen branches in the way. But over time, the way is cleared and a path forms---not through any centrally coordinated effort, but simply as a result of all the individuals walking that way day after day.

The cleared path is the product of labor---not any individual’s labor, but all of them together. If one villager decided to take advantage of the now-created path by setting up a gate and charging tolls, he would be violating the collective property right that the villagers together have earned.

Public property can also be the product of gift. In 19th-century England, it was common for roads to be built privately and then donated to the public for free use. This was done not out of altruism but because the roadbuilders owned land and businesses alongside the site of the new road, and they knew that having a road there would increase the value of their land and attract more customers to their businesses. Thus, the unorganized public can legitimately come to own land, both through original acquisition (the mixing of labor) and through voluntary transfer.
\end{quote}

So Long provides two cases that he sees as legitimate group property: (1) where a group communally “mix their labour” with an object in nature, and (2) where a man transfers ownership of his private property to a group in common. The issue with (1) is that Long relies on the faulty labour theory of property. It is not mixing labour with land which imbues a man with ownership, as we have seen it is the nature of scarcity giving rise to the potential for conflict which implies property rights. To demonstrate the failure of this theory more thorougly, allow me to quote Kinsella at length:\footnote{N. Stephan Kinsella, \emph{Against Intellectual Property}, pp. 36–38}
\begin{quote}
As noted before, some libertarian IP advocates, such as Rand, hold that creation is the source of property rights. This confuses the nature and reasons for property rights, which lie in the undeniable fact of scarcity. Given scarcity and the correspondent possibility of conflict in the use of resources, conflicts are avoided and peace and cooperation are achieved by allocating property rights to such resources. And the purpose of property rights dictates the nature of such rules. For if the rules allocating property rights are to serve as objective rules that all can agree upon so as to avoid conflict, they cannot be biased or arbitrary. For this reason, unowned resources come to be owned—homesteaded or appropriated—by the first possessor.

The general rule, then, is that ownership of a given scarce resource can be identified by determining who first occupied it. There are various ways to possess or occupy resources, and different ways to demonstrate or prove such occupation, depending upon the nature of the resource and the use to which it is put. Thus, I can pluck an apple from the wild and thereby homestead it, or I can fence in a plot of land for a farm. It is sometimes said that one form of occupation is “forming” or “creating” the thing. For example, I can sculpt a statue from a block of marble, or forge a sword from raw metal, or even “create” a farm on a plot of land.

We can see from these examples that creation is relevant to the question of ownership of a given “created” scarce resource, such as a statue, sword, or farm, only to the extent that the act of creation is an act of occupation, or is otherwise evidence of first occupation. However, “creation” itself does not justify ownership in things; it is neither necessary nor sufficient. One cannot create some possibly disputed scarce resource without first using the raw materials used to create the item. But these raw materials are scarce, and either I own them or I do not. If not, then I do not own the resulting product. If I own the inputs, then, by virtue of such ownership, I own the resulting thing into which I transform them.

Consider the forging of a sword. If I own some raw metal (because I mined it from ground I owned), then I own the same metal after I have shaped it into a sword. I do not need to rely on the fact of creation to own the sword, but only on my ownership of the factors used to make the sword. And I do not need creation to come to own the factors, since I can homestead them by simply mining them from the ground and thereby becoming the first possessor. On the other hand, if I fashion a sword using your metal, I do not own the resulting sword. In fact, I may owe you damages for trespass or conversion.
\end{quote}

Of course, both of Long's cases, that of group ``labour mixing'' and that of transferring a private property right to a collective fail on the grounds that they do not resolve the contradiction---we still have it that the subset deemed unworthy of control do not actually own the property in question. For his second case in particular, Long is making the error of placing contract theory at the root, rather than property theory. But contracts are contracts \emph{about} property---a contract defines a set of transfers of title to property. The concept of ``contract'' is \emph{descendant from}, not antecedent to the concept of ``property.'' This is the stolen concept fallacy---Long steals the concept of contract in his attempted inversion.\footnote{See Nathaniel Branden, \emph{The Fallacy of The Stolen Concept}}

So to summarise: our rational theory of law has it that property rights are \emph{private} rights of \emph{exclusive}\footnote{In the sense of \emph{excluding} people from the property.} control over \emph{scarce} resources, whose purpose is to avoid conflicts over said resources.
\end{document}
